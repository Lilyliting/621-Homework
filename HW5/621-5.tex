\documentclass{article} 
\usepackage{geometry}
\usepackage{url}
\geometry{left=3.5cm, right=3.5cm, top=2.5cm, bottom=2.5cm}
\usepackage{graphicx} 
\usepackage[round]{natbib}
\bibliographystyle{plainnat}
\usepackage{amsmath,amsthm}
\usepackage{amsmath}
\usepackage{mhchem}

\begin{document}  
\title{621-Homework 5}
\author{Liting Hu}

\maketitle

\section{Problem 1.}
\subsection{}
Implement a basic Monte Carlo using m simulation trials for European Call and Put options:
\begin{verbatim}
function [Price, SD, SE, Time] = Monte_Carlo(iscall, S0, K, Tm, r, sigma, div, n, m)
% iscall=1: european call option; iscall=0: put option
% S0: Initial stock price
% K: Strike price
% Tm: Time to maturity
% r: Interest rate
% sigma: Volatility
% delta: Continuous dividend rate
% n: Time steps
% m: Quantity of trials
tic;
dt = Tm/n;
S0 = ones(1,m).*S0;
lnS_T = log(S0);

nudt = (r-div-sigma^2/2)*dt;
sigsdt = sigma*sqrt(dt);

for i=1:n
    epsilon = symbol*normrnd(0,1,1,m);
    dS = nudt+sigsdt.*epsilon; % evolve the stock price
    lnS_T = lnS_T+dS;
end
S_T = exp(lnS_T);
 % Stock prices at maturity

if iscall==1
    profits = max(S_T-K,0)*exp(-r*Tm);
else
    profits = max(K-S_T,0)*exp(-r*Tm);
end

Price = mean(profits);
SD = std(profits); % Standard deviation
SE = SD/sqrt(m); % Standard error
Time=toc;
\end{verbatim}
The answer will be indicated in next subsection.

\subsection{}

Define a Monte Carlo function using the antithetic variates method:
\begin{verbatim}
function [Price, SD, SE, Time]=Monte_Carlo_AVR(iscall, S0, K, Tm, r, sigma, div, n, m)
% Monte Carlo valuation with Antithetic Variance reduction
% iscall=1: european call option; iscall=0: put option
% S0: Initial stock price
% K: Strike price
% Tm: Time to maturity
% r: Interest rate
% sigma: Volatility
% delta: Continuous dividend rate
% n: Time steps
% m: Quantity of trials
tic;
dt = Tm/n;
S0 = ones(1,m).*S0;
ST1 = Path_to_T(1, S0, r, sigma, div, dt, n, m);
ST2 = Path_to_T(-1, S0, r, sigma, div, dt, n, m);

if iscall==1
    profits = (max(ST1-K,0)+max(ST2-K,0))/2*exp(-r*Tm);
else
    profits = (max(K-ST1,0)+max(K-ST2,0))/2*exp(-r*Tm);
end

Price = mean(profits);
SD = std(profits);
SE = SD/sqrt(m);
Time=toc;
\end{verbatim}

Using the delta?based control variate:
\begin{verbatim}
function [Price, SD, SE, Time]=Monte_Carlo_DC(iscall, S0, K, Tm, r, sigma, div, beta, n, m)
% Monte Carlo valuation with delta control
% iscall=1: european call option; iscall=0: put option
% S0: Initial stock price
% K: Strike price
% Tm: Time to maturity
% r: Interest rate
% sigma: Volatility
% div: Continuous dividend rate
% n: Time steps
% m: Quantity of trials
tic;

dt = Tm/n;
nudt = (r-div-sigma^2/2)*dt;
sigsdt = sigma*sqrt(dt);
erddt = exp((r-div)*dt);

S0 = ones(1,m).*S0;

St = S0;
cv = zeros(1,m);


for i=1:n
    t=(i-1)*dt;
    d1 = (log(St/K)+(r-div+sigma^2/2)*(Tm-t))/sigma/sqrt(Tm-t);
    if iscall==1 % Black-Scholes Formulas for Delta
        delta = exp(-div*(Tm-t))*normpdf(d1,0,1);
    else
        delta = exp(-div*(Tm-t))*(normpdf(d1,0,1)-1);
    end
    epsilon = normrnd(0,1,1,m);
    Stn = St.*exp(nudt+sigsdt.*epsilon); % evolve the stock price
    cv = cv+delta.*(Stn-St*erddt);
    St = Stn;
end

if iscall==1
    profits = max(St-K,0)+beta*cv;
else
    profits = max(K-St,0)+beta*cv;
end

Price = mean(profits)*exp(-r*Tm);
SD = std(profits);
SE = SD/sqrt(m);
Time=toc;
\end{verbatim}

Using the antithetic variates method and delta?based control variate:
\begin{verbatim}
function [Price, SD, SE, Time]=Monte_Carlo_AVRDC(iscall, S0, K, Tm, r, sigma, div, beta, n, m)
% Monte Carlo valuation with Antithetic Variance reduction and delta control
% iscall=1: european call option; iscall=0: put option
% S0: Initial stock price
% K: Strike price
% Tm: Time to maturity
% r: Interest rate
% sigma: Volatility
% div: Continuous dividend rate
% n: Time steps
% m: Quantity of trials
tic;

dt = Tm/n;
nudt = (r-div-sigma^2/2)*dt;
sigsdt = sigma*sqrt(dt);
erddt = exp((r-div)*dt);

S0 = ones(1,m).*S0;

St1 = S0;
St2 = S0;
cv1 = zeros(1,m);
cv2 = zeros(1,m);

for i=1:n
    t=(i-1)*dt;
    d1 = (log(St1/K)+(r-div+sigma^2/2)*(Tm-t))/sigma/sqrt(Tm-t);
    d2 = (log(St2/K)+(r-div+sigma^2/2)*(Tm-t))/sigma/sqrt(Tm-t);
    if iscall==1 % Black-Scholes Formulas for Delta
        delta1 = exp(-div*(Tm-t))*normpdf(d1,0,1);
        delta2 = exp(-div*(Tm-t))*normpdf(d2,0,1);
    else
        delta1 = exp(-div*(Tm-t))*(normpdf(d1,0,1)-1);
        delta2 = exp(-div*(Tm-t))*(normpdf(d2,0,1)-1);
    end

    epsilon = normrnd(0,1,1,m);
    Stn1 = St1.*exp(nudt+sigsdt.*epsilon); % evolve the stock price
    Stn2 = St2.*exp(nudt-sigsdt.*epsilon);
    cv1 = cv1+delta1.*(Stn1-St1*erddt);
    cv2 = cv2+delta2.*(Stn2-St2*erddt);
    St1 = Stn1;
    St2 = Stn2;
end

if iscall==1
    profits = (max(St1-K,0)+beta*cv1 + max(St2-K,0)+beta*cv2)/2;
else
    profits = (max(K-St1,0)+beta*cv1 + max(K-St2,0)+beta*cv2)/2;
end

Price = mean(profits)*exp(-r*Tm);
SD = std(profits);
SE = SD/sqrt(m);
Time=toc;
\end{verbatim}

All the results are showed in the table below:

\begin{table}[h]
\begin{tabular}{|c|c|c|c|c|}

\hline
& MC & MC.AVC & MC.DC & MC.AVCDC \\
\hline
Price & 9.1349 & 9.1385 & 9.1296 & 9.1334 \\
\hline
SD & 13.6937 & 9.6744 & 11.0993 & 7.8494 \\
\hline
SE & 0.0137 & 0.0097 & 0.0111 & 0.0078 \\
\hline
Time(s) & 6.8481 & 13.8185 & 13.9291 & 21.9580 \\
\hline

\end{tabular}
\end{table}


\section{Problem 2.}
\subsection{}

Apply the Euler discretization schemes and implement all the five schemes listed.
\begin{verbatim}
function [Price, Bias, RMSE, Time]=MC_Heston(Scheme, S0, K, V0, Tm, r, sigma, kappa, theta, rho, n, m)
% For european call option
% S0: Initial stock price
% K: Strike price
% V0: Initial volatility
% Tm: Time to maturity
% r: Interest rate
% sigma: Volatility variance
% kappa: speed of mean-reversion of the variance
% theta: the long-term average variance
% rho: Correlation coefficient between two Brownian motions
% n: Time steps
% m: Quantity of trials

tic;
dt = Tm/n;
St = ones(1,m).*S0;
lnSt = log(St);
Vt1 = ones(1,m).*V0;
dWv = normrnd(0,1,n,m)*sqrt(dt);
dZ = normrnd(0,1,n,m)*sqrt(dt);
dWs = rho*dWv+sqrt(1-rho^2)*dZ;
    
for i=1:n
    if Scheme==1 || Scheme==4 || Scheme==5
        Vt2 = max(Vt1,0); % Effective variance
    else
        Vt2 = abs(Vt1);
    end
    
    lnSt = lnSt+(r-Vt2/2)*dt+sqrt(Vt2).*dWs(i,:);
    if Scheme==1
        f1=max(Vt1,0);
        f2=f1;
        f3=f1;
    elseif Scheme==2
        f1=abs(Vt1);
        f2=f1;
        f3=f1;
    elseif Scheme==3
        f1=Vt1;
        f2=f1;
        f3=abs(Vt1);
    elseif Scheme==4
        f1=Vt1;
        f2=f1;
        f3=max(Vt1,0);
    else
        f1=Vt1;
        f2=max(Vt1,0);
        f3=f2;
    end

    Vt1 = f1+kappa*(theta-f2)*dt+sigma.*f3.^(1/2).*dWv(i,:);
    


end
St = exp(lnSt);

Calls = max(St-K, 0);
Price = mean(Calls)*exp(-r*Tm);
Bias = abs(Price-6.8061);
RMSE = rms(Calls-6.8061);

Time = toc;
\end{verbatim}

All the results are showed in the table below: (Quantity of trials is 1e6.)

\begin{table}[h]
\begin{tabular}{|c|c|c|c|c|cl}

\hline
Scheme & Absorption & Reflection & Higham and Mao & Partial truncation & Full truncation\\
\hline
Price & 6.8812 &  6.9201 & 6.8567 & 6.8285 & 6.7791 \\
\hline
Bias & 0.0751 & 0.1140 & 0.0506 & 0.0224 & 0.0270\\
\hline
RMSE & 7.8474 & 7.9933 & 7.7839 & 7.6809 & 7.6613\\
\hline
Time(s) & 2.6324 & 2.7505 & 2.8136 & 2.7355 &  2.6148\\
\hline

\end{tabular}
\end{table}

\section{Problem 3.}
\subsection{}
By function chol in matlab, the lower triangular matrix L is easy to calculate.
\begin{verbatim}
A=[1,0.5,0.2;0.5,1,-0.4;0.2,-0.4,1];
L = chol(A,'lower');
\end{verbatim}

\subsection{}

To calculate all m paths, define function Correlated\_BM:
\begin{verbatim}
function [St]=Correlated_BM(S0,Tm,dt,A,sigma,MU,m)
% S0: Initial stock prices
% Tm: Time to maturity
% dt: Time intervel
% A: Correlation matrix
% sigma: Volatilities
% MU: Risk free rates
% m: Quantity of trials

L = chol(A,'lower'); % Cholesky factorization

n = Tm/dt;
nudt = (MU-0.5*sigma.^2)'*dt*ones(1,m);
St = zeros(n+1,m,3);
St(1,:,1)=S0(1);
St(1,:,2)=S0(2);
St(1,:,3)=S0(3);
lnSt = log(St);

for i=1:n
    Zt = normrnd(0,1,3,m)*sqrt(dt);
    Wt = L*Zt;
    for j=1:3
        lnSt(i+1,:,j)=lnSt(i,:,j)+nudt(j,:)+Wt(j,:).*sigma(j);
    end
end

St = exp(lnSt);
\end{verbatim}

And then, draw a plot:
\begin{verbatim}
S0 = [100,101,98];
MU = [0.03,0.06,0.02];
sigma = [0.05,0.2,0.15];

Tm = 100/365;
m = 1000;
dt = 1/365;

St = Correlated_BM(S0,Tm,dt,A,sigma,MU,m);
Onepath = St(:,1,:);
Onepath = reshape(Onepath,(Tm/dt+1)*1,3);
plot(Onepath)
legend('Stock1','Stock2','Stock3')
\end{verbatim}

\begin{figure}[h] 
\begin{center} 
\includegraphics[width = 15cm]{3-1.png}  
\caption{} 
\end{center} 
\end{figure}


\subsection{}

For a simple average basket:
\begin{verbatim}
function [Price]=Basket_option(iscall,S0,K,Tm,dt,A,sigma,MU,m)
% iscall=1: european call option; iscall=0: put option
% S0: Initial stock prices
% K: Strike price
% Tm: Time to maturity
% dt: Time intervel
% A: Correlation matrix
% sigma: Volatilities
% MU: Risk free rates
% m: Quantity of trials

St = Correlated_BM(S0,Tm,dt,A,sigma,MU,m);

Ut = mean(St,3); % a simple average basket
UT = Ut(end,:);
Price = mean(max((-1)^iscall*(K-UT),0));
\end{verbatim}

The price of this basket call option is 2.0076.

\subsection{}

For this exotic option:
\begin{verbatim}
function [Price]=Basket_exotic_option(S0,K,B,Tm,dt,A,sigma,MU,m)
% For european call option
% S0: Initial stock prices
% K: Strike price
% B: Barrier
% Tm: Time to maturity
% dt: Time intervel
% A: Correlation matrix
% sigma: Volatilities
% MU: Risk free rates
% m: Quantity of trials

St = Correlated_BM(S0,Tm,dt,A,sigma,MU,m);
UT = zeros(1,m); % Define option price matrix
for i=1:m
    pathi = St(:,i,:);
    pathi = reshape(pathi,(Tm/dt+1)*1,3);
    
    if max(pathi(:,2)) > B % condition (i)
        UT(i) = max(pathi(end,2)-K,0);
        continue;
    elseif max(pathi(:,2)) > max(pathi(:,3)) % condition (ii)
        UT(i) = max(pathi(end,2)-K,0)^2;
        continue;
    elseif mean(pathi(:,2)) > mean(pathi(:,3)) % condition (iii)
        UT(i) = max(mean(pathi(:,2))-K,0);
        continue;
    else % condition (iv)
        ST = mean(pathi(end,:));
        UT(i) = max(ST-K,0);
    end
end

Price = mean(UT)*exp(-mean(MU)*Tm);
\end{verbatim}

Price of this exotic option is 5.8203.



\# All the matlab code can be found in zip file.


%\begin{figure}[htbp] 
%\begin{center} 
%\includegraphics[width=0.47\textwidth]{crustalsplit.png}  
%\caption{The effects of shear wave splitting in the Moho P to S converted phase. Top shows a schematic seismogram in the fast/slow coordinate system with split horizontal Ps components.(cited from: McNamara and Owens, 1993)} 
%\label{crustalspliting} 
%\end{center} 
%\end{figure}

\end{document}